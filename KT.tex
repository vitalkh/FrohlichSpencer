\documentclass[11pt,reqno]{article}


% fix the Q letter $Q(\varrho)=0$


% Packages
\usepackage{amsmath, amsthm, amsopn, amssymb}
\usepackage{enumerate, enumitem, tikz, etoolbox, intcalc, geometry, caption, subcaption, afterpage}
\usepackage[english]{babel}
\usepackage{mathrsfs, mathtools}


\DeclarePairedDelimiter\abs{\lvert}{\rvert}%
\DeclarePairedDelimiter\norm{\lVert}{\rVert}%


% % Margins
%\setlength{\topmargin}{0in}
\setlength{\leftmargin}{0in}
\setlength{\rightmargin}{0in}
\setlength{\evensidemargin}{0in}
\setlength{\oddsidemargin}{0in}

% Text area size
\setlength{\textwidth}{6.55in}
\setlength{\textheight}{9.06in}


% Definitions

\newtheorem{thm}{Theorem}[section]
\newtheorem{lemma}[thm]{Lemma}
\newtheorem{prop}[thm]{Proposition}
\newtheorem{definition}[thm]{Definition}
\newtheorem{cor}[thm]{Corollary}
\newtheorem{claim}[thm]{Claim}
\newtheorem{quest}[thm]{Question}
\newtheorem{fact}[thm]{Fact}
\theoremstyle{definition}
\newtheorem*{remark}{Remark}
\newtheorem*{remarks}{Remarks}
\newtheorem*{conj}{Conjecture}

\newcommand{\twopartdef}[4]
{
	\left\{
		\begin{array}{ll}
			#1 & #2 \\
			#3 & #4
		\end{array}
	\right.
}
\newcommand{\threepartdef}[6]
{
	\left\{
		\begin{array}{lll}
			#1 & #2 \\
			#3 & #4 \\
			#5 & #6
		\end{array}
	\right.
}

\numberwithin{equation}{section}

\begin{document}
\section{Introduction}\label{sec:introduction}
\newcommand*{\mida}{d\mu_{\text{\tiny$\beta$}}(\phi)}
\newcommand*{\midar}{d\mu_\beta(\phi)}


\begin{thm}
This is a new thm.
\end{thm}

\begin{thm}[brackets]
This is a new thm.
\end{thm}
Proof: Done!

Add $a$ squared and $b$ squared to get $c$ squared. Or, using a more mathematical approach 
\begin{equation} 
a^2 + b^2 = c^2 
\end{equation} 
Einstein says 
\begin{equation} \label{clever}
E = mc^2 
\end{equation} 
He didn’t say 
\begin{equation} 
1 + 1 = 3 \tag{dumb}
\end{equation} 

This is a reference to \eqref{clever}.

\section{Chapter2}\label{sec:chap2}
\newcommand*{\ens}{\mathscr{N}}



\section{Chapter3}\label{sec:chap3}
Let $\mathscr{ABCDEFGHIJKLMNOPQRSTUVWXYZ}$

\subsection{Notation}
\newcommand*{\SqrsSet}{\mathscr{S}_k(\varrho)}
\newcommand*{\SqrsSetTag}{\mathscr{S}^\prime_k(\varrho)}
\newcommand*{\SqrsSetDtag}{\mathscr{S}^{\prime\prime}_k(\varrho)}
\newcommand*{\SqrsSetGamma}{\mathscr{S}_{\gamma(k)}(\varrho)}
\newcommand*{\SqrsSetTagGamma}{\mathscr{S}^\prime_{\gamma(k)}(\varrho)}
\newcommand*{\SqrsSetDtagGamma}{\mathscr{S}^{\prime\prime}_{\gamma(k)}(\varrho)}
First, denote $b = \frac{\alpha}{2} + 1 + \log_2(M)$. Also, denote $\gamma(k)=[\alpha^{-1}(k-b-2)]$, where $[x]$ denotes the integer part of x. 

Let $\SqrsSet$ be a minimal collection of $2^k\times2^k$ squares covering the support of $\varrho$. By minimal we mean that $\SqrsSet$ is chosen such that its cardinality, $\abs{\SqrsSet}$, is minimal, i.e. 
$$\abs{\SqrsSet} = A_k(\varrho)$$
Now we define $\SqrsSetTag$ and $\SqrsSetDtag$.
First, in the case where $\SqrsSet$ consists of only one square and $Q(\varrho)=0$, we set $\SqrsSetTag=\emptyset$ and $\SqrsSetDtag=\emptyset$.
In any other case, we define $\SqrsSetTag$ to be the sub-collection of those squares $s^\prime$ in $\SqrsSet$ which are far separated from other squares in $\SqrsSet$, in the sense that $$dist(s^\prime, s)\geq 2M2^{\alpha k + \alpha /2}\equiv2^{\alpha k + b}$$
for all $s\in\SqrsSet$ ($s\neq s^\prime$), and define $\SqrsSetDtag := \SqrsSet \setminus \SqrsSetTag$.\\
Denote also 
\begin{equation} \label{label:ATagDef}
A^\prime(\varrho) = A_0(\varrho) + \sum_{k=1}^\infty \abs{\SqrsSetTag}
\end{equation}

\begin{remarks}
\begin{enumerate}
\item For $k > log_2 d(\varrho)$, $\SqrsSet$ consists of a single square covering all of $supp(\varrho)$. Hence, if $Q(\varrho) = 0$, we get $\SqrsSetTag=\emptyset$, so that the sum \eqref{label:ATagDef} terminates at some $k=k(\varrho) \leq log_2d(\varrho)$ (and if $\varrho$ is charged, we get $k(\varrho)=\infty$, $\abs{\SqrsSetTag} = 1$, for all $k > log_2 d(\varrho)$).
\item We note that, by (???? Thm2.1), $s\cap\varrho$ is charged for each $s\in\SqrsSetTag$. This fact will be used later, for the proof of (???????).
\end{enumerate}
\end{remarks}



\subsection{Upper bound on $A(\varrho)$}
\begin{prop} \label{prop:ABound}
\begin{equation} \label{label:eq31}
A(\varrho) \leq D_2 A^\prime(\varrho)
\end{equation}
for some finite constant $D_2$ independent of $\varrho$ and $\mathscr{N}$.
\end{prop}


\begin{lemma} \label{lemma:GammaReason}
\begin{equation} \label{label:GammaRec}
A_k(\varrho) \leq \frac{1}{2} A_{\gamma(k)}(\varrho) + \abs{\SqrsSetTagGamma}
\end{equation}
for k such that $\abs{\SqrsSetGamma} \geq 2$.
\end{lemma}

\begin{proof}
Let k be such that $\abs{\SqrsSetGamma} \geq 2$. We claim that $[m/2]$ squares of size $2^k$ are enough to cover $\SqrsSetDtagGamma$, where $m=\abs{\SqrsSetDtagGamma}$.\\
Let $s_1 \in \SqrsSetDtagGamma$, and let $s_2 \in \SqrsSetDtagGamma$ such that $$dist(s_1, s_2) < 2^{\alpha \gamma(k) + b} \leq 2^{k-b-2 + b} = 2^{k-2}.$$
Since $d(s_1)=d(s_2)=\sqrt{2} 2^{\gamma(k)}$ and $\gamma(k) \leq k-4$ (since $\gamma(k) < k-3$ for $b>1$, $\alpha > 1$), 
$$d(s_1) + d(s_2) + dist(s_1, s_2) \leq 2^{\gamma(k) + 2} + 2^{k-2} < 2^{k-1},$$
and thus $s_1, s_2$ can be covered by a single square in $\SqrsSet$.\\
Next, let $s_1, \cdots, s_t$, $t \geq 3$ be squares in $\SqrsSetDtagGamma$ such that
$$dist(s_l, s_{l+1}) < 2^{\alpha \gamma(k) + b}, \qquad l=1,\cdots,t$$
Then $s_{l-1}, s_l, s_{l+1}$ can be covered by a single $2^k\times2^k$ square, for any $l=2,3,\cdots,t-1$, since
$$d(s_{l-1}) + d(s_l) + d(s_{l+1}) + dist(s_{l-1}, s_l) + dist(s_l, s_{l+1}) \leq 2^{\gamma(k) + 3} + 2^{k-1} < 2^k.$$
Therefore, [m/2] $2^k\times2^k$ squares suffice to cover $s_1, \cdots, s_t$ for $t=2,3,\cdots$. So given a cover $\SqrsSetGamma$ of $supp\varrho$, we can replace the squares in $\SqrsSetDtagGamma$ with $[m/2]$ squares of size $2^k\times2^k$, and resize the squares in $\SqrsSetTagGamma$ to be squares of size $2^k\times2^k$, and get a cover of $supp\varrho$ with squares of size $2^k\times2^k$, of size $[m/2] + \abs{\SqrsSetTagGamma}$, and therefore
\begin{equation} \label{label:lab38}
A_k(\varrho) = \abs{\SqrsSet} \leq
\frac{1}{2}\abs{\SqrsSetDtagGamma} + \abs{\SqrsSetTagGamma} \leq
\frac{1}{2} A_{\gamma(k)}(\varrho) + \abs{\SqrsSetTagGamma}
\end{equation}
Note that \eqref{label:lab38} holds only if $\abs{\SqrsSetGamma} \geq 2$, since otherwise $\SqrsSetGamma = \SqrsSetTagGamma \cup \SqrsSetDtagGamma$ isn't true for $Q(\varrho)=0$.
\end{proof}

\begin{lemma} \label{lemma:GammaCond}
$\abs{\SqrsSetGamma} \geq 2$ holds for every $k \leq n(\varrho) \leq log_2(Md(\varrho)^\alpha) + 1$.
\end{lemma}
\begin{proof}
For such k, 
$$\gamma(k) \leq \alpha^{-1}(k-b-2) \leq
\alpha^{-1}(log_2(M) + \alpha log_2d(\varrho) + 1) - \alpha^{-1}(\alpha/2 + 1 + log_2(M)) - 2\alpha^{-1} \leq $$
$$log_2d(\varrho) - 2\alpha^{-1} - 1/2 < 
log_2d(\varrho)-1/2.$$
Therefore $\sqrt{2}\cdot2^{\gamma(k)} < d(\varrho)$, and so more than one square of size $2^{\gamma(k)}\times2^{\gamma(k)}$ is needed to cover $supp(\varrho)$, and so $\abs{\SqrsSetGamma} > 1$ must hold.
\end{proof}

\begin{proof}[Proof of Proposition \eqref{prop:ABound}]
Let $\delta:=b-2$. Clearly, inequality \eqref{label:GammaRec} can only be applied if $\gamma(k) \geq 0$, i.e. $k \geq \delta$. For each $k$, we now iterate inequality \eqref{label:GammaRec} $l=l(k)$ times, where $l(k)$ is the maximal number for which $\gamma^{l(k)}(k) \geq 0$. Here $\gamma^m$ denotes the m-fold composition of $\gamma$ with itself.\\
Using \eqref{label:GammaRec} iteratively yields
\begin{equation} \label{label:lab310}
A_k(\varrho) \leq 
2^{-l}A_{\gamma^l(k)}(\varrho) + \sum_{m=0}^{l-1} 2^{-m} \abs{\mathscr{S}^\prime_{\gamma^{m+1}(k)}(\varrho)} \leq 
2^{-l}A_0(\varrho) + \sum_{m=0}^{l-1} 2^{-m} \abs{\mathscr{S}^\prime_{\gamma^{m+1}(k)}(\varrho)}
\end{equation}
with $l=l(k)$, and using $A_0(\varrho) \geq A_j(\varrho)$, for all $j \geq 0$.\\

We start by estimating $l(k)$. For this purpose, we extend the definition of $\gamma$ to the whole interval $[\delta, \infty)$, by setting $\gamma(x)=[\alpha^{-1}(x-\delta)]$, for every $x \geq \delta$. Obviously, $\gamma$ is monotone increasing, and
$$\gamma(x) > \alpha^{-1}(x-\delta)-1.$$
These two properties yield
$$\gamma^2(x) = \gamma(\gamma(x)) \geq \gamma(\alpha^{-1}(x-\delta)-1) > 
\alpha^{-2}x - \alpha^{-2}\delta - \alpha^{-1}\delta - \alpha^{-1} - 1,$$
and, by induction (assuming that we stay in the interval $[\delta, \infty]$),
\newcommand*{\alphadelta}{(\alpha-1)^{-1}(\alpha+\delta)}
$$
\gamma^m(x) > \alpha^{-m}x - \delta (\sum_{j=1}^m \alpha^{-1}) - \sum_{j=0}^{m-1}\alpha^{-j} \geq
$$
$$
\geq \alpha^{-m}x - (\delta\alpha^{-1}+1)(\sum_{j=0}^\infty\alpha^{-j}) = 
\alpha^{-m}x - \alphadelta.
$$
For $k \geq \alphadelta$, we define $l_0(k)$ to be the maximal integer for which
\begin{equation} \label{label:eq313}
\alpha^{-l_0(k)}x - \alphadelta \geq 0.
\end{equation}
Then $\gamma^{l_0(k)}(k) > 0$ and therefore $l(k) \geq l_0(k)$. Set $k_0=(\alpha-1)^{-1}(\alpha+\delta)$, and so by taking logarithm in \eqref{label:eq313} we obtain
\begin{equation} \label{label:eq314}
l(k) \geq \twopartdef { 0 } {0 \leq k < k_0} {{[\frac{log_2(k/k_0)}{log_2(\alpha)}]}} {otherwise}
\end{equation}\\

Next, we estimate the cardinality $\abs{N_{m,j}}$ of the sets
$$
N_{m,j} := \{k: \gamma^m(k) = j\}
$$
As before, we use the monotonicity of $\gamma$ together with $\gamma(x) \leq \alpha^{-1}(x-\delta)$ to obtain
$$
\gamma^2(x) \leq \gamma(\alpha^{-1}(x-\delta)) \leq \alpha^{-2}x - \alpha^{-2}\delta - \alpha^{-1}\delta 
$$
and again, by induction.
$$
\gamma^m(x) \leq \alpha^{-m}-\delta(\sum_{l=1}^m\alpha^{-l})
$$
Next, let $k_-$ be the minimal and $k_+$ the maximal integer in $N_{j,m}$ (the maximum exists for every $j,m$, since $\gamma^m(k) > \alpha^{-m}k - k_0 > j$ for every $k>k_1$ for some $k_1$). Then
$$
\alpha^{-m}k_+ - \delta(\sum_1^ma^{-l})-\sum_0^{m-1}\alpha^{-l} \leq \gamma^m(k_+) = j = \gamma^m(k_-) \leq \alpha^{-m}k_- - \delta(\sum_1^m\alpha^{-l})
$$
and therefore
\begin{equation} \label{label:lab316}
\abs{N_{m,j}} = k_+ - k_- + 1 \leq \alpha^m \sum_0^{m-1}\alpha^{-l} + 1 \leq \alpha^m \sum_0^{\infty}\alpha^{-l} + 1 = \alpha^m \frac{\alpha}{\alpha-1} + 1 \leq \alpha^m \frac{2\alpha}{\alpha-1}
\end{equation}
for $\alpha > 1$.\\

Now, using \eqref{label:lab310} we have
\begin{equation} 
A(\varrho) = \sum_{k=0}^{n(\varrho)}A_k(\varrho) \leq 
\sum_{k=0}^{n(\varrho)} 2^{-l(k)}A_0(\varrho) + 
\sum_{k=0}^{n(\varrho)}\sum_{m=0}^{l(k)-1} 2^{-m} \abs{\mathscr{S}^\prime_{\gamma^{m+1}(k)}(\varrho)} .
\end{equation}

We continue by bounding both terms. First note that by \eqref{label:eq314}
$$
\sum_{k=0}^{n(\varrho)} 2^{-l(k)} \leq \sum_{k=0}^{\infty} 2^{-l(k)} \leq k_0 + \sum_{k=k_0+1}^{\infty} 2^{-l(k)} \leq k_0 + \sum_{k=k_0+1}^{\infty}(k_0/k)^{1/log_2(\alpha)} \equiv E < \infty
$$
for $1 < \alpha < 2$, i.e. $1/log_2\alpha > 1$. \\
Let $k(\varrho)$ be the largest k for which $\SqrsSetTag \neq \emptyset$, so using \eqref{label:lab316}
$$
\sum_{k=0}^{n(\varrho)}\sum_{m=0}^{l(k)-1} 2^{-m} \abs{\mathscr{S}^\prime_{\gamma^{m+1}(k)}(\varrho)} = 
$$
$$
=\sum_{j=0}^{k(\varrho)} \bigg( \sum_{k=0}^{n(\varrho)}\sum_{m=0}^{l(k)-1} 2^{-m} \delta_{\gamma^{m+1}(k) = j} \bigg) \abs{\mathscr{S}^\prime_{j}(\varrho)} \leq
$$
$$
\leq \sum_{j=0}^{k(\varrho)} \bigg( \sum_{m,k=0}^{\infty} 2^{-m} \delta_{\gamma^{m+1}(k) = j} \bigg) \abs{\mathscr{S}^\prime_{j}(\varrho)} = 
$$
$$
\leq \sum_{j=0}^{k(\varrho)} \bigg( \sum_{m=0}^{\infty} 2^{-m} \abs{N_{m+1,j}} \bigg) \abs{\mathscr{S}^\prime_{j}(\varrho)} \leq
$$
$$
\leq \sum_{j=0}^{k(\varrho)} \bigg( \alpha \frac{2\alpha}{\alpha-1} \sum_{m=0}^{\infty} \big( \frac{\alpha}{2} \big) ^m \bigg) \abs{\mathscr{S}^\prime_{j}(\varrho)} \leq
$$
$$
\leq F\sum_{j=0}^{\infty} \abs{\mathscr{S}^\prime_{j}(\varrho)}
$$
where F is some constant such that $\alpha \frac{2\alpha}{\alpha-1} \sum_{m=0}^{\infty} \big( \frac{\alpha}{2} \big) ^m \leq F$, and assuming $1 < \alpha < 2$.\\
The proposition now follows by setting $D_2 = max(E,F)$.
\end{proof}


\section{Renormalization of Multipole Densities}\label{sec:chap4}



\subsection{Main Result}
The main result of the this section is
\begin{thm} \label{thm:thm41}
Let $\ens$ be an ensemble satisfying properties (a)-(c) of Theorem (????????). For each $\varrho \in \ens$, let $\sigma(\varrho)$ be some real number. Then
\begin{equation}
\int\prod_{\varrho \in \ens}[1 + K(\varrho)cos(\phi(\varrho) + \sigma(\varrho))]\mida = 
\int\prod_{\varrho \in \ens}[1 + z(\beta,\varrho)cos(\phi(\bar{\varrho}) + \sigma(\varrho))]\mida, 
\end{equation}
where
$$
z(\beta,\varrho) = K(\varrho)exp[-\beta E_{loc}(\varrho, \ens)]
$$
and $E_{loc}(\varrho, \ens)$ is some function of $\varrho$ satisfying 
\begin{equation}
E_{loc}(\varrho, \ens) \geq D_1 A^\prime(\varrho).
\end{equation}
Moreover, $\bar{\varrho}$ is a charge density which depends linearly on $\varrho$ and satisfies
\begin{enumerate}[label={\alph*)}]
\item $Q(\bar{\varrho}) = Q(\varrho)$, and
\item $d(\bar{\varrho}) \leq 2d(\varrho)$ (unless $\varrho$ is charged, in which case $d(\bar{\varrho}) = \infty, E_{loc}(\varrho, \ens) = \infty$).
\end{enumerate}
\end{thm}

\subsection{The Charged Density}

\subsection{First Renormalization Step}

We start by decomposing the support of a charge density $\varrho \in \ens$ into two disjoint subsets $\Omega_1, \Omega_2$, such that no two sites in $\Omega_1$ are nearest neighbours (for example choose it to be sites of $supp\varrho$ on two complementary grids which cover $\mathbb{Z}^2$). Let $\varrho_l(j) = \varrho(j)$, for $h \in \Omega_l$, $\varrho(j)=0$ otherwise, $l=1,2$. Clearly $\varrho_1 + \varrho_2 = \varrho$. We define
$$
R_0\varrho(j) := \varrho_2(j) + 1/4\sum_{\abs{k-j}=1}\varrho_1(k).
$$
Also, denote $E_0(\varrho) = 1/8\sum\varrho(j)^2$. By an appropriate choice of $\Omega_1$ we have
\begin{equation} \label{label:lab419}
E_0(\varrho_1) \geq \frac{1}{2} E_0(\varrho).
\end{equation}

\begin{remarks}
It's easy to see that
\begin{enumerate}
\item $R_0\varrho(j) = 0$ for $j\in\Omega_1$.
\item $Q(R_0\varrho) = Q(\varrho)$.
\end{enumerate}
\end{remarks}


\begin{prop}[First Renormalization Step] \label{prop:lem418}
Using the notation above
$$
\int \prod [1+K(\varrho)cos(\phi(\varrho) + \sigma(\varrho))] \mida = 
\int \prod [1+K(\varrho)e^{-\beta E_0(\varrho_1)}cos(\phi(R_0\varrho) + \sigma(\varrho))] \mida
$$
\end{prop}


\begin{lemma} \label{lemma:contour}
\begin{equation}
\int e^{i q x} I_\beta(x)dx = \int e^{i q (x+ia)} I_\beta(x+ia)dx
\end{equation}
where $I_\beta(x) = exp[-(\beta/2)\phi^2]$, and $a\in \mathbb{R}$.
\end{lemma}
\begin{proof}
Let $b \in (0,\infty)$. First note that 
$$
\abs{I_\beta(b+ix)} = \abs{exp(-(\beta/2) (b+ix)^2} = \abs{I_\beta(b)}exp[(\beta/2)x^2].
$$
By integrating on a closed contour, we obtain
$$
\int_{-b \rightarrow b} e^{i q \omega} I_\beta(\omega)d\omega =
\int_{-b+ia \rightarrow b+ia} e^{i q \omega} I_\beta(\omega)d\omega + 
$$
\begin{equation} \label{label:contour}
\int_{-b \rightarrow -b+ia} e^{i q \omega} I_\beta(\omega)d\omega +
\int_{b+ia \rightarrow b} e^{i q \omega} I_\beta(\omega)d\omega.
\end{equation}
where we can bound the last two terms by
$$
\Bigg| \int_0^a e^{iq(b+ix)}I_\beta(b+ix)dx \Bigg| \leq 
\int_0^a \Big| e^{iq(b+ix)}I_\beta(b+ix)\Big| dx \leq
$$
$$
\leq max(1, e^{-qa}) I_\beta(b) \int_0^a exp[(\beta/2)x^2] \leq const(a)I_\beta(b)
$$
Therefore taking $b\rightarrow\infty$ in \eqref{label:contour} yields the lemma.
\end{proof}

\begin{lemma} \label{lemma:lem42}
If $G(\phi)$ is a functional independent of $\phi(j_0)$ then
\begin{equation}
\int e^{i q \phi(j_0)} G(\phi)\mida = e^{-\beta q^2/8} \int e^{i q \bar{\phi}(j_0)} G(\phi)\mida
\end{equation}
where $\bar{\phi}(j_0)=1/4\sum_{\abs{k-j_0}=1} \phi(k)$.
\end{lemma}
\begin{proof}
The proof follows by explicit integration of the $\phi(j_0)$ variable, using lemma \eqref{lemma:contour}:
$$
\int e^{i q \phi(j_0)} exp \Big[ -\frac{1}{2\beta} \sum_{\abs{k-j_0}=1}(\phi(j_0)-\phi(k))^2 \Big] d\phi(j_0) =
$$
$$
\int e^{i q \phi(j_0) - q^2\beta/4} exp \Big[ -\frac{1}{2\beta} \Big( \sum_{\abs{k-j_0}=1}(\phi(j_0)-\phi(k))^2 \Big) - i q \phi(j_0) + iq\bar{\phi}(j_0) + \frac{q^2\beta}{8}\Big] d\phi(j_0)
$$
\end{proof}

\begin{proof}[Proof of proposition \eqref{prop:lem418}]
The proof follows by applying lemma \eqref{lemma:lem42} iteratively.
First note that for M large enough, $dist(\varrho, \varrho^\prime) \geq 3$ for all $\varrho, \varrho^\prime$ in $\ens$, so that $supp(R_0\varrho) \cap supp(R_0\varrho^\prime) = \emptyset$, for any choice of $\Omega_l(\varrho), \Omega_l(\varrho^\prime)$, $l=1,2$. Thus $\Omega_l(\varrho)$ can be chosen independently of $\Omega_l(\varrho^\prime)$ for $\varrho \neq \varrho^\prime$ such that \eqref{label:lab419} holds for all $\varrho \in \ens$.\\
Now, for every $\varrho$, we apply lemma \eqref{lemma:lem42} iteratively over $j \in \Omega_1$, with $q=\varrho_1(j)$, and obtain
$$
\int e^{i\phi(\varrho)}\mida = 
\int e^{i\phi(\varrho_2)}exp \Big[ {-\frac{\beta}{8}\sum_{j \in \Omega_1}\varrho_1^2(j)} + i\sum_{j \in \Omega_1} \bar{\phi}(j)\varrho_1(j)\Big] \mida = 
$$
$$
=\int e^{-\beta E_0(\varrho_1)} exp \Big[ \sum_{j \in \Omega_2}\phi(j) (\varrho_2(j) + 1/4\sum_{\abs{k-j}=1} \varrho_1(k) )\Big] \mida =
\int e^{-\beta E_0(\varrho_1)}e^{\phi(R_0\varrho)}\mida
$$
where we used the fact that no two sites in $\Omega_1$ are nearest neighbours, and the equality
$$
\sum_{l\in\Omega_1}\bar{\phi}(l)\varrho_1(l) = 
\sum_{l\in\Omega_1} \frac{1}{4} \sum_{\substack{\abs{t-l}=1 \\ (i.e.\ t\in \Omega_2)}} \phi(t)\varrho_1(l) = 
\sum_{t \in \Omega_2} \phi(t) \frac{1}{4} \sum_{\abs{t-l}=1} \varrho_1(l).
$$
The lemma now follows from 
$$
1+K(\varrho)cos(\phi(\varrho) + \sigma(\varrho)) = 
1 + 
\frac{K(\varrho)}{2}e^{i\sigma(\varrho)}e^{i\phi(\varrho)} + 
\frac{K(\varrho)}{2}e^{-i\sigma(\varrho)}e^{-i\phi(\varrho)}.
$$
\end{proof}






\subsection{Main Renormalization Step}
\subsubsection{Additional Notation and Proposition Statement}
Let $D(\varrho)$ be all those sites of $\mathbb{Z}^2$ contained in an open disc in $\mathbb{R}^2$ of radius $2d(\varrho)$ centred at a point $x(\varrho)$ such that $supp\varrho \subset D(\varrho)$ and
$$
dist(supp\varrho, \partial D(\varrho)) = d(\varrho)
$$
where $\partial D(\varrho)$ is the outer boundary of $D(\varrho)$ [we can take for example $x(\varrho)$ to be one of the points which satisfy the minimum in $d(\varrho)$].\\
Now let us fix a particular charge density $\varrho^* \in \ens$, with $Q(\varrho^*)=0$.
%TODO: maybe we can stop caring here about the charge, as it must be 0 if we already eliminated the charged density
We define 
$$
\ens^* = \{\varrho: \varrho \in \ens, d(\varrho)\leq 2d(\varrho^*),\ \varrho \neq \varrho^*,\  Q(\varrho)=0\}
$$
We consider a mapping, G, from $\ens^*$ into functions of the variables $\{ \phi(j) \}$, with the properties
\begin{enumerate}
\item $G(\phi, \varrho)$ only depends in the variables $\{ \phi(j): j \in D(\varrho) \}$.
\item 
\begin{equation} \label{lab:eq428}
G(\phi + b, \varrho) = G(\phi, \varrho), 
\end{equation}
for any function $b$ on $\mathbb{Z}^2$ taking an arbitrary, constant value, $r$, on $D(\varrho)$.
\end{enumerate}

\begin{remark}
$G$ has the form 
$$G(\phi, \varrho) = 1+K(\varrho)cos(\phi(\varrho) + \sigma(\varrho)),\ or\ =1+z(\beta, \varrho)cos(\phi(\bar{\varrho}) + \sigma(\varrho))
$$
\end{remark}
We now define
$$
F_{\ens^*}(\phi) = \prod_{\varrho \in \ens^*} G(\phi, \varrho)
\prod_{\varrho^* \neq \varrho \in \ens \setminus \ens^*}[1+K(\varrho)cos(\phi(\varrho) + \sigma(\varrho))].
$$
Let $\Omega_1$ be the set of all sites contained in a disc of radius $R_1 \leq d(\varrho^*)$ in $\mathbb{R}^2$ centred at some point $x_1$, with
$$
\Omega_1 \subset D(\varrho^*), \ \ supp\varrho^* \cap \Omega_1 \neq \emptyset.
$$
Let $\Omega_2$ be some other subset of $D(\varrho^*)$, disjoint from $\Omega_1$, such that
$$
dist(x_1, \partial\Omega_2) \geq RR_1, \ \ where \ (1 < R) \ and\ (RR_1 \leq diam(D(\varrho^*))).
$$ 
Let $f_j$ be some linear transformations of $\varrho^*$, with $suppf_j \subset \Omega_j$, $j=1,2$, such that also $\sum_{k \in \Omega_i}f_j(k) = \sum_{k \in \Omega_i}\varrho^*(k)$, and denote
\begin{equation} \label{label:qdef}
q := \sum_{k \in \Omega_1}\varrho^*(k) = \sum_{k \in \Omega_1}f_1(k).
\end{equation}
Also denote 
$$
i_\beta(\phi, a) = exp\Big(-\frac{i}{\beta} \sum_{\substack{\abs{k-j}=1 \\ dist(j,x_1) \leq RR_1}} (\phi(k)-\phi(j)) \cdot (a(k)-a(j))\Big).
$$
Note that $i_\beta(\phi+b, a) = i_\beta(\phi, a)$ for any $b$ which is constant on $\{j: dist(j,x_1) \leq RR_1\} $.

$$
C_1(\phi, \varrho^*) = e^{i\phi(f_1)} e^{i\sigma(\varrho^*)}
$$ 
$$
C_2(\phi, \varrho^*) = e^{i\phi(f_2)}
$$ 

\begin{prop}[Main Renormalization Step] \label{prop:prop44}
Under the notations above, for $\alpha > 3/2$ and M sufficiently large:
\begin{equation} \label{label:eq44}
\int e^{i\phi(f_1)} e^{i\phi(f_2)} e^{i\sigma(\varrho^*)} F_{\ens^*}\mida = 
e(C_1, \beta) \int i_\beta(\phi,a) e^{i\phi(f_1)} e^{i\phi(f_2)} e^{i\sigma(\varrho^*)} F_{\ens^*}\mida
\end{equation}
where
\begin{enumerate}[label={\alph*)}]
\item $e$ is a numerical factor satisfying
$$
0 \leq e(f_1, \beta) \leq exp\Big[ -ln(R)\frac{q^2\beta}{2d(M)} \Big]
$$
for some constant $d(M)$, with $d(M) \rightarrow 24\pi$, as $M\rightarrow \infty$, and
\item $i_\beta(\phi,a)$ only depends on the variables $\{ \phi(j):j \in R\Omega_1 \}$, with $R\Omega_1 := \{ j: dist(j,x_1) \leq RR_1 \}$, where the function $a$ is defined in \eqref{label:eq437}, with:
\begin{enumerate}
\item $a(j)=0$ for $dist(j,x_1) \geq RR_1$, and
\item $a$ is constant on $\Omega_1$.
\end{enumerate}

\end{enumerate}
\end{prop}
\subsubsection{Connected Components of $\bigcup_{\varrho \in \ens^*}D(\varrho)$}
Let $\ens^\prime$ be some sub-ensemble of $\ens$, and let $D_i$ be the connected components of $\bigcup_{\varrho \in \ens^\prime}D(\varrho)$ (we say that $D(\varrho_1), D(\varrho_2)$ are connected if $D(\varrho_1)\cap D(\varrho_2)\neq\emptyset$). We express each $D_i$ as a union of $D(\varrho)$, $\varrho \in S_i \subset \ens^\prime$. For each $S_i$, denote by $\varrho_i \in S_i$ a density with maximal diameter.
\begin{remark}
The reason for looking on connected components, is that later, in the proof of proposition \eqref{prop:prop44}, we will make a change of variables $\phi(j) \longmapsto \phi(j) + ia(j)$, and we would want $a(j)$ to be constant on every $D(\varrho), \varrho \neq \varrho^*$ .
\end{remark}

\begin{lemma} \label{lemma:appendixE1}
For M large enough, $diam(D_i) \leq 5 d(\varrho_i)$, and also $\varrho_i$ is a unique density with maximal diameter.
\end{lemma}

\begin{remark}
From now on, we will associate $D_i$ with it's unique density with maximal diameter, $\varrho_i$.
\end{remark}

\begin{proof}
We will show that there is only one density $\varrho_i \in S_i$ with maximal diameter, and that all other densities $\varrho \in S_i$ have a diameter which is $\ll d(\varrho_i)$, for $M$ large enough.\\
We shall prove this by induction on the size of $d(\varrho_i)$, where $\varrho_i$ is a density with maximal diameter. First, if $d(\varrho_i) \leq 2$, and $M$ large enough, $D_i = D(\varrho_i)$, and therefore $diam(D_i) = diam(D(\varrho_i)) \leq 5d(\varrho_i)$ (given any $\varrho^\prime$ with $d(\varrho^\prime) \leq 2$, $dist(\varrho_i,\varrho^\prime) \geq M$, and so for $M>20$, $D(\varrho_i)$ and $D(\varrho^\prime)$ are far enough from each other, to not be in the same connected component).\\
Our induction step will be increasing $d(\varrho_i)$ by a factor of 2. Let
$$
\mathscr{M} = \{ \varrho \in S_i: d(\varrho) \geq \frac{1}{2}d(\varrho_i) \}
$$
and define $D_{ij}$ to be the connected components of $ \bigcup_{\varrho \in S_i \setminus \mathscr{M}} D(\varrho)$. By induction, 
$$
diamD_{ij} \leq 5d(\varrho_{ij}) \leq \frac{5}{2} d(\varrho_i).
$$
Since $D_{ij}$ intersect $D(\varrho)$ for some $\varrho \in \mathscr{M}$, in some point $y$, we conclude that
$$
Md(\varrho_{ij})^\alpha \leq dist(\varrho, \varrho_{ij}) \leq 
dist(\varrho, y) + dist(y, \varrho_{ij}) \leq 4d(\varrho) + 5d(\varrho_{ij}) \leq 8d(\varrho_i)
$$
and thus for $M$ large enough $d(\varrho_{ij}) \ll d(\varrho_i)$, so
$$
diamD_{ij} \leq 5d(\varrho_{ij}) \leq \frac{1}{2} d(\varrho_i).
$$
Now, for $M$ large enough, any two densities $\varrho_1 \neq \varrho_2$ in $\mathscr{M}$, are far enough from each other, so that $D(\varrho_1)$ and $D(\varrho_2)$ are not connected, since if there exists $y \in D(\varrho_1)\cap D(\varrho_2)$ then
$$
M\Big( \frac{d(\varrho_i)}{2} \Big)^\alpha \leq dist(\varrho_1, \varrho_2) \leq dist(\varrho_1, y) + dist(y, \varrho_2) \leq 4d(\varrho_1) + 4d(\varrho_2) \leq 8d(\varrho_i)
$$
which is false for $M$ large enough.\\
So if $\mathscr{M}$ has two element, $\varrho_1, \varrho_2$, then $D(\varrho_1), D(\varrho_2)$ are connected by some $D_{ij}$, and so:
$$
M\Big( \frac{d(\varrho_i)}{2} \Big)^\alpha \leq dist(\varrho_1, \varrho_2) \leq dist(\varrho_1, y_1) + dist(y_1, y_2) + dist(y_2, \varrho_2) \leq 8d(\varrho_i) + diam(c_{ij}) \leq 9d(\varrho_i)
$$
where $y_1 \in D_{ij} \cap D(\varrho_1), y_2 \in D_{ij} \cap D(\varrho_2)$, which is a false for $M$ large enough.\\
Therefore $\abs{\mathscr{M}} = 1$ for $M$ large enough, which implies that there is a unique density $\varrho_i \in S_i$ with maximal diameter. Also, all of the $D_{ij}$ intersect $D(\varrho_i)$. Thus,
$$
diam(D_i) \leq 4d(\varrho_i) + 2max(diam(D_{i,j})) \leq 5d(\varrho_i)
$$
\end{proof}

\begin{lemma} \label{lemma:appendixE2}
For M large enough, $\abs{\partial D_i} \leq 20 d(\varrho_i)$.
\end{lemma}

\begin{proof}
We shall prove this lemma by induction on $d(\varrho_i)$ as in lemma \eqref{lemma:appendixE1}. So again, for $M$ large enough, $\abs{S_i} = 1$, and then
$$
\abs{\partial D_i} = \abs{\partial D(\varrho_i)} \leq 2\pi\cdot 2d(\varrho_i) < 20d(\varrho_i)
$$
Now, using the same notation as in lemma \eqref{lemma:appendixE1}, $\abs{\partial D_{ij}} \leq 20 d(\varrho_{ij})$, and hence
$$
\abs{\partial D_i} \leq \abs{\partial D(\varrho_i)} + \sum_j \abs{\partial D_{ij}} \leq
4\pi d(\varrho_i) + 20 \sum_j d(\varrho_{ij})
$$
Now we bound the number of components $D_{ij}$ with $2^{k-1} \leq diam(D_{ij}) \leq 2^k$ [which intersect $\partial D(\varrho_i)$]. Let $D_{i1}, D_{i2}$ with such diameter, so
$$
M2^{\alpha(k-1)} \leq dist(\varrho_{i1}, \varrho_{i2}) \leq 5d(\varrho_{i1}) + dist(D_{i1}, D_{i2}) + 5d(\varrho_{i2}) \leq dist(D_{i1}, D_{i2}) + 10 \cdot 2^k
$$
and thus for M large enough we obtain $\frac{M}{8}2^{\alpha k} \leq dist(D_{i1}, D_{i2})$. Therefore, the number of possible $D_{ij}$ in this scale, which intersect $\partial D(\varrho_i)$ is bounded by
$$
\frac{\abs{\partial D(\varrho_i)}}{\frac{M}{8}2^{\alpha k}} \leq \frac{const}{M} 2^{-\alpha k} d(\varrho_i),
$$
and so
$$
20 \sum d(\varrho_{ij}) \leq \sum_{k=1}^\infty \frac{const}{M} 2^{-\alpha k} \cdot 2^k \leq \frac{const^\prime}{M} d(\varrho_i) \leq 2d(\varrho_i)
$$
for $M$ large enough. And so we obtain $\abs{\partial D_i} \leq 20d(\varrho_i)$.
\end{proof}

\subsubsection{Proof of Proposition \eqref{prop:prop44}}
\begin{proof}[Proof of Proposition \eqref{prop:prop44}]
For notational simplicity. we suppose the $\Omega_1$ is centred at the origin, i.e., $x_1=0$. Let b(j) be a real-valued function on $\mathbb{R}^2$ defined by
$$
b(j) = 
\threepartdef{lnR,} {\abs{j} \leq R_1}
{ln[RR_1\abs{j}^{-1}],}{R_1 \leq \abs{j} \leq RR_1}
{0,}{\abs{j} \geq RR_1.}
$$
We make the following change of variable:
$$
\phi(j) \longmapsto \phi(j) + ia(j),
$$
where $a$ is a real-valued function defined in terms of $b$, using $D_i$, the connected components of $\bigcup_{\varrho \in \ens^*}D(\varrho)$:
\begin{equation} \label{label:eq437}
a(j) = 
\twopartdef{\gamma b(j),} {j \notin a}
{\gamma b(x_i),}{j \in D_i}
\end{equation}
where $x_i$ is a point of $D_i$ chosen so that if $D_i$ meets $\abs{k} \leq R_1$, $a(j)\equiv \gamma ln(R)$, and if $D_i$ meets $\abs{k} \geq RR_1$, $a(j)\equiv 0$ for all $j \in D_i$. Otherwise we set $x_i=x(\varrho_i)$, the center of $D(\varrho_i)$. The constant $\gamma$ will be chosen later.

\begin{remarks}
Note that:
\begin{enumerate}
\item For M large enough, $D_i$ can not meet both $\abs{k} \leq R_1$, and $\abs{k} \geq RR_1$. The idea is that if $D_i$ meets $\abs{k} \leq R_1$, then $D_i$ is very close to $\varrho^*$ (up to $R_1$), and so it must be small compared to $R_1$. 
To be exact, suppose that it isn't true, then there are $\abs{y_1} \leq R_1, \abs{y_2} \geq RR_1$, $y_i \in D_i$. Let $z_1 \in \varrho^*$ with $\abs{z_1} \leq R_1$, and $z_2 \in \varrho_i \subseteq D_i$, then
$$
M\frac{d(\varrho_i)}{2} \leq M\Big( \frac{d(\varrho_i)}{2} \Big)^\alpha \leq dist(\varrho^*, \varrho_i) \leq dist(z_1, z_2) \leq dist(z_1, y_1) + dist(y_1, y_2) + dist(y_2, z_2) \leq 
$$
$$
\leq 2R_1 + diam(D_i) + diam(D_i) \leq 2R_1 + 10 d(\varrho_i),
$$
where the last inequality follows from \eqref{lemma:appendixE2}. Therefore $\frac{M-20}{2}d(\varrho_i) \leq 2R_1$, and so
$$
diam(D_i) \leq 5d(\varrho_i) \leq \frac{20}{M-20} R_1 < (R-1)R_1
$$
for $M$ large enough, and $R>????????????????????????????$.
%TODO: make sure to bound R from below. the bound doesn't have to be sqrt(2), but it has to be constant > 1. if not - this remark is false.
\item The choice of a(j) is such that the change of variable will be constant in $D(\varrho)$ for every $\varrho \neq \varrho^*$.\\
\end{enumerate}
\end{remarks}

Now consider the affects of the change of variables:
\begin{enumerate}[label={\alph*)}]
\item $F_{\ens^*}$ is unaffected. If $\varrho \in \ens^*$, by property \eqref{lab:eq428}, $G(\phi+ia, \varrho) = G(\phi, \varrho)$, as $a$ is constant on $D(\varrho)$. If $\varrho \notin \ens^*$, i.e. $d(\varrho) > 2d(\varrho^*)$, then
$$
dist(\varrho, \varrho^*) \geq Md(\varrho^*)^\alpha \geq Md(\varrho^*) \geq \frac{M}{4}RR_1
$$
and so for $M$ large enough, $a(j)=0$ for $j \in supp(\varrho)$.
\item $e^{i\sigma(\varrho^*)}$ is unaffected by the change, as it doesn't depend on $\phi$.
\item $ e^{i\phi(f_2)}$ is unaffected by the change, as it depends only on $\phi(j), j\in \Omega_2$, and $a(j)=0$ for such $j$ (if $j \in D_i$ for some $i$, then $x_i$ is chosen such that $a(j)=0$, and otherwise $b(j)=0$).
\item 
$$
\mida(\phi + ia) = i_\beta(\phi, a) \cdot exp\Big[\frac{1}{2\beta} \norm{\nabla a}_2^2 \Big]\mida(\phi)
$$
\item $e^{i(\phi + ia)(f_1)} = e^{-\gamma (lnR) q} e^{i\phi(f_1)}$, as $a(j)=\gamma ln(R)$ for $\abs{j} \leq R_1$, i.e. $j \in \Omega_1$, and $f_1$ depends only on $j \in \Omega_1$ ($q = \sum_{k \in \Omega_1}f_1(k)$ as defined in \eqref{label:qdef}).
\end{enumerate}
Setting 
$$
e(f_1, \beta) = e^{-\gamma (lnR) q} exp\Big[\frac{1}{2\beta} \norm{\nabla a}_2^2 \Big]
$$
gives us \eqref{label:eq44}.

It remains to prove the bound on $e(f_1, \beta)$, which is really the essential part of the proposition.

By the definition of $a$
\begin{equation} \label{label:eq442}
\norm{\nabla a}_2^2 = 
\sum_{\substack{\abs{k-j}=1 \\ \abs{j} \leq RR_1}}(a(k)-a(j))^2 =
\gamma^2 \sum_{\substack{\abs{k-j}=1 \\ \abs{j} \leq RR_1 \\ j,k \notin \cup_{\varrho \in \ens^*} D(\varrho)}}(b(k)-b(j))^2 + \sum_i B_i
\end{equation}
where
$$
B_i = \gamma^2 \sum_{j \in \partial D_i} (b(x_i) -b(j))^2.
$$
The first term on the right side of \eqref{label:eq442} is bounded above by
$$
\gamma^2 \sum_{\substack{\abs{k-j}=1 \\ \abs{j} \leq RR_1}} (b(k)-b(j))^2 =
\gamma^2 \sum_{\substack{\abs{k-j}=1 \\ R_1 \leq \abs{j} \leq RR_1}} (ln\abs{j} - ln\abs{k})^2 \leq
$$
\begin{equation} \label{label:eq444}
\leq \gamma^2 \sum_{r = R_1}^{RR_1} (2\pi r) \cdot 4 \cdot (ln(r+1)-ln(r))^2 \ \leq \ 
\gamma^2 8\pi \sum_{r = R_1}^{RR_1} \frac{1}{r} \ \leq \ \gamma^2 24\pi lnR
\end{equation}

Note that $B_i=0$, unless 
\begin{equation} \label{label:eq446}
R_1-6d(\varrho_i) \overset{(i)}{\leq} \abs{x_i} \overset{(ii)}{\leq} RR_1 + 6d(\varrho_i),
\end{equation}

otherwise $D_i$ lies entirely outside the annulus $R_1 \leq \abs{j} \leq RR_1$, hence $b(x_i)=b(j)$ for $j \in \partial D_i$.
Fix such i. 

\begin{lemma}
This lemma contains several calculations, which basically show that $d(\varrho_i)$ is small compared to $R_1$ for $M$ large enough:
\begin{enumerate}
\item 
\begin{equation} \label{label:eq448}
\frac{M-12}{2}d(\varrho_i)^\alpha \leq \abs{x_i}
\end{equation}
\item 
\begin{equation} \label{label:eq450}
(1-\delta) R_1 \leq \abs{x_i} \leq RR_1(1+\delta),
\end{equation}
for some $\delta = \delta(M,\alpha)$ which tends to 0 as $M \rightarrow \infty$.
\item 3
\begin{equation} \label{label:eq451}
\abs{x_i} - 6d(\varrho_i) \geq \frac{1}{2} \abs{x_i}
\end{equation}
\end{enumerate}
\end{lemma}
\begin{proof}

Denote $d = min(d(\varrho_i), d(\varrho^*))$, so by (???????2.4) $ Md^\alpha \leq dist(\varrho_i, \varrho^*) $. Note that
$$
dist(\varrho^*, \varrho_i) \leq dist(\varrho^*, 0) + dist(0, x_i) + dist(x_i, \varrho_i) \leq R_1 + \abs{x_i} + 6d(\varrho_i)
$$
and using (i), (ii), $\varrho_i \in \ens^*$, and $RR_1 \leq diam(D(\varrho^*))$, we obtain
$$
Md^\alpha \leq dist(\varrho^*, \varrho_i) \leq R_1 + \abs{x_i} + 6d(\varrho_i) \overset{(iii)}{\leq} 2\abs{x_i} + 12d(\varrho_i) \overset{(iv)}{\leq} 2RR_1 + 24 d(\varrho_i) \leq 
$$
$$
\leq 2RR_1 + 48 d(\varrho^*)  \leq 56d(\varrho^*).
$$
Therefore, for $\alpha > 1$ and $M$ large enough $d=d(\varrho_i)$ holds. We then get from (iii) and (iv) that \eqref{label:eq448} holds.
and %TODO: why??
$$
\frac{M-24}{2}d(\varrho_i)^\alpha \leq RR_1.  
$$
Combining this with \eqref{label:eq446}, we obtain
$$
R_1 - \frac{12}{M-12} \abs{x_i} \leq \abs{x_i} \leq RR_1 + \frac{12}{M-12}\abs{x_i}
$$
and so \eqref{label:eq450} holds.
Moreover, for $M$ large enough, we obtain \eqref{label:eq451} from \eqref{label:eq448}.
\end{proof}


Next, let $\ens_k^* = \{ \varrho_i \in \ens^* : 2^k \leq d(\varrho_i) < 2^{k+1} , \eqref{label:eq450}\ holds\ for\ \varrho_i \}$.
\begin{lemma}
$$
\sum_{\varrho_i \in \ens_k^*} \abs{x_i}^{-2} \leq K_2 \frac{2^{-2\alpha k}}{M^2} lnR,
$$
for some constant $K_2$, independent of $\varrho_i, \alpha, M$.
\end{lemma}
\begin{proof}
Let $\varrho_1, \varrho_2 \in \ens_k^*$, let $x(\varrho_1), x(\varrho_2)$ be the the centres of $D(\varrho_1), D(\varrho_2)$, and let $c_k:=\frac{M-12}{2}2^{\alpha k}$, then
$$
dist(\varrho_1, \varrho_2) \leq dist(\varrho_1, x(\varrho_1)) + dist(x(\varrho_1), x(\varrho_2)) + dist(x(\varrho_2), \varrho_2) \leq 
$$
$$
\leq d(\varrho_1) + \abs{x(\varrho_1) - x(\varrho_2)} + d(\varrho_2)
$$
and so using (??????????????2.4)
$$
\abs{x(\varrho_1) - x(\varrho_2)} \geq  dist(\varrho_1, \varrho_2) - d(\varrho_1) - d(\varrho_2) \geq M2^{\alpha k} - 2^{k+2} \geq 2c_k.
$$
Also, using \eqref{label:eq448}, 
$$
\abs{x_i} \geq \frac{M-12}{2} d(\varrho_i)^\alpha \geq c_k .
$$
Now we bound the number of $\varrho_i \in \ens_k^*$ with $rc_k \leq \abs{x_i} \leq rc_{k+1}$, by $K_1r$ for $K_1$ independent of $\varrho_i, \alpha, M$, provided $M$ is sufficiently 
large. The main observation is that for two such $\varrho_i$, $\abs{x(\varrho_i) - x(\varrho_j)}$ is bounded below, so that around every $x(\varrho_1)$ there's a disc of radius $2c_k$ with no other $x_j$. Now we can bound the number of such discs which can be inside the annulus $rc_k \leq \abs{x_i} \leq rc_{k+1}$ by
$$
\frac{\pi(r+1)^2 c_k^2 - \pi r^2 c_k^2}{\pi 4 c_k^2} \leq K_1r.
$$
By \eqref{label:eq450}, a valid value for $r$ must satisfy
$$
rc_k \leq RR_1(1+\delta)
\text{ and }
(r+1) c_k \geq (1-\delta) R_1
$$
so that the range of values of r is given by $m_1 \leq r \leq m_2$, where
$$
m_1 := max(c_k^{-1} (1-\delta)R_1 - 1, 1)
$$
$$
m_2 := max(c_k^{-1} (1+\delta)RR_1, 1)
$$
Now,
$$
\sum_{\varrho_i \in \ens_k^*} \abs{x_i}^{-2} \leq 
\sum_{r=m_1}^{m_2} r^{-2}c_k^{-2} \cdot K_1r \leq 
K_1^\prime \frac{2^{-2\alpha k}}{M^2} \sum_{r=m_1}^{m_2} r^{-1} \leq 
$$
$$
\leq K_2 \frac{2^{-2\alpha k}}{M^2} lnR
$$
\end{proof}

Now let us go back to bounding $\sum B_i$





\end{proof}





\subsection{Proof of Theorem \eqref{thm:thm41}}

\end{document}
